\documentclass[12pt]{article}
\renewcommand{\figurename}{Fig}
\usepackage{xiaolab}
\usepackage[dvipdfmx]{graphicx}
\usepackage{tocloft}
\usepackage{here}
\usepackage{amsmath,amssymb,mathtools} %数学記号の記述に使用
\usepackage{algorithm,algpseudocode} %アルゴリズムの記述に使用
\usepackage[dvipdfmx]{color} %文字の色付けに使用
\usepackage{booktabs}
\usepackage{bm}
\usepackage{url}
\usepackage{lscape} % 図表を横向きにするために使用
\usepackage{arydshln}
\usepackage[hang,small,bf]{caption}
\usepackage[subrefformat=parens]{subcaption}
\usepackage[titletoc,title]{appendix}
\captionsetup{compatibility=false}

\begin{document}
\pagestyle{empty}

%目次
\clearpage
\begin{appendices}
  \makeatletter
    \renewcommand{\theequation}{% 式番号の付け方
    \thesection.\arabic{equation}}
    \@addtoreset{equation}{section}

    \renewcommand{\thefigure}{% 図番号の付け方
    \thesection.\arabic{figure}}
    \@addtoreset{figure}{section}

    \renewcommand{\thetable}{% 表番号の付け方
    \thesection.\arabic{table}}
    \@addtoreset{table}{section}

    \renewcommand{\thealgorithm}{
    \thesection.\arabic{algorithm}}
    \@addtoreset{algorithm}{section}
  \makeatother
  \floatname{algorithm}{}
  \input{appendix_A}
\end{appendices}

\end{document}
